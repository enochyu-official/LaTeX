\documentclass{article}
\usepackage{graphicx}
\usepackage{amsmath,amsthm,amssymb}
\usepackage[font=small,labelfont=bf]{caption}
\usepackage{tikz}
\usetikzlibrary{calc, angles, quotes, shapes.geometric}
\usepackage{tkz-euclide}
\usepackage[inline]{asymptote}
\usepackage{float}
\usepackage[margin=1in]{geometry}
\usepackage{gensymb}
\usepackage{hyperref}
\hypersetup{
    colorlinks=true,
    linkcolor=blue,
    filecolor=magenta,      
    urlcolor=cyan,
    pdftitle={Overleaf Example},
    pdfpagemode=FullScreen,
    }
\usepackage{fancyhdr}
\pagestyle{fancy}
\fancyhead[R]{Enoch Yu}
\pagenumbering{gobble}
\usepackage{enumitem}
\newtheorem{theorem}{Theorem}[section]
\newtheorem{lemma}[theorem]{Lemma}
\newtheorem*{lemma*}{Lemma}
\newtheorem{sublemma}{Lemma}[section]
\newtheorem{proposition}{Proposition}
\newtheorem{corollary}{Corollary}[theorem]
\newenvironment{solution}{\begin{trivlist}\item[]{\bf Solution}}{\qed \end{trivlist}}
\newcommand{\verteq}{\rotatebox{90}{$\;\;=\;\;$}}
\newcommand*\circled[1]{\tikz[baseline=(char.base)]{
            \node[shape=circle,draw,inner sep=1pt] (char) {#1};}}
\newcommand{\triangled}[1]{\tikz[baseline=(char.base)]{
            \node[shape=regular polygon, regular polygon sides=3, draw, inner sep=0.2pt] (char) {#1};}}

\title{Problem Set 17}
\author{Enoch Yu}
\date{June 2025}

\begin{document}

\section*{Irreducible Polynomials}
\[
F(x)=f(x)\cdot Q(x)+R(x)
\]
$\text{The Degree of $R(x)$} < \text{The Degree of $f(x)$}$ ALWAYS!!
\[
\text{Often manipulated as }\frac{F(x)-R(x)}{f(x)}=Q(x).
\]

\section*{Problem}
Let $a>0$, and let $P(x)$ be a polynomial with integer coefficients such that $P(1)=P(3)=P(5)=P(7)=a$, and $P(2)=P(4)=P(6)=P(8)=-a$. What is the smallest possible value of $a$?
\begin{solution}
\\\\
\textbf{Key Word} Irreducible Polynomials
\\\\
Hmm.. What can we do? \\
First, let's try writing the equations.
\begin{align*}
    P(x) &= (x-1)(x-3)(x-5)(x-7)Q(x)+a \\
    P(x) &= (x-2)(x-4)(x-6)(x-8)Q'(x)-a
\end{align*}
Now, why don't we try substituting diverse $x$ values in an equation?
\begin{align*}
    P(1) &= (-1)(-3)(-5)(-7)Q'(1)-a=a \\
    P(3) &= (1)(-1)(-3)(-5)Q'(3)-a=a \\
    P(5) &= (3)(1)(-1)(-3)Q'(5)-a=a \\
    P(7) &= (5)(3)(1)(-1)Q'(7)-a=a
    \\\\
    \therefore 105Q'(1) &= -15Q'(3) = 9Q'(5) = -15Q'(7) = 2a
\end{align*}
Notice that $a$ must be a positive integer because $P(x)$ consists of integer coefficients!! In another words, the possible values for $2a$ are the multiples of $LCM(105, 15, 9)$, which is 315. However, because $a$ must be an integer, the smallest possible value of $a$ is $\boxed{315}$.
\end{solution}

\newpage
\section*{Problem}
Let $P(x)$ be a polynomial with integer coefficients such that $P(1)=P(3)=P(5)=P(7)=315$, and $P(2)=P(4)=P(6)=P(8)=-315$. What is $P(x)$ with the smallest leading degree?
\begin{solution}
\\\\
\textbf{Key Word} Irreducible Polynomials
\\\\
First, let's try writing an equation for $P(x)$.
\[
P(x)=(x-2)(x-4)(x-6)(x-8)Q(x)-315
\]
Next, why don't we substitute known values?
\begin{align*}
    P(1) &= 105Q(1) = 630 \\
    P(3) &= -15Q(3) = 630 \\
    P(5) &= 9Q(5) = 630 \\
    P(7) &= -15Q(7) = 630 \\
\end{align*}
Here, we know that $Q(3)=Q(7)=-42$. Therefore, the following equation could be written.
\[
Q(x) = (x-3)(x-7)Q'(x)-42
\]
By utilizing $P(1)=630$ and $P(5)=630$, we could further gain information.
\begin{align*}
    105(12Q'(1)-42) &= 630 \\
    9(-4Q'(5)-42) &= 630
    \\\\
    Q'(1) &= 4 \\
    Q'(5) &= -28
\end{align*}
In order for $P(x)$ to have least degree, $Q'(x)$ must have least degree. In other words, $Q'(x)=-8x+12$.
\[
\therefore \boxed{P(x)=(x-2)(x-4)(x-6)(x-8)\{(x-3)(x-7)(-8x+12)-42\}-315}
\]
\end{solution}

\newpage
\section*{Problem}
For certain real numbers $a$, $b$, and $c$, the polynomial
\[
g(x)=x^3+ax^2+x+10
\]
has three distinct roots, and each root of $g(x)$ is also a root of the polynomial
\[
f(x)=x^4+x^3+bx^2+100x+c.
\]
What is $f(1)$?
\begin{solution}
\\\\
\textbf{Key Word} Vieta's Formula
\\\\
The problem told us that the following expression is true.
\[
f(x)=g(x)\cdot(x-r_4)
\]
Moreover, using Vieta's Formula, we can get our information!
\begin{align*}
    r_1+r_2+r_3 &= -a \\
    r_1r_2+r_1r_3+r_2r_3 &= 1 \\
    r_1r_2r_3 &= -10
    \\\\
    r_1+r_2+r_3+r_4 &= -1 \\
    r_1r_2+\dots+r_3r_4 &= b \\
    r_1r_2r_3+\dots+r_2r_3r_4 &= -100 \\
    r_1r_2r_3r_4 &= c
\end{align*}
Using the information above, let's try writing $b$, $c$, and $r_4$ in different ways.
\begin{align*}
    b &= 1-ar_4 \\
    c &= -10r_4 \\
    r_4 &= a-1 \\
    r_4 &= -90\ (\because r_1r_2r_3+r_4(r_1r_2+r_1r_3+r_2r_3)=-10+r_4=-100
    \\\\
    \therefore a&=-89, b=-8009, c=900
    \\\\
    f(1)&=1+1+b+100+c=\boxed{-7007}
\end{align*}
\end{solution}

\end{document}
