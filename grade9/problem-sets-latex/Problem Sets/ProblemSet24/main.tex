\documentclass{article}
\usepackage{graphicx}
\usepackage{amsmath,amsthm,amssymb}
\usepackage[font=small,labelfont=bf]{caption}
\usepackage{tikz}
\usepackage{pgfplots}
\pgfplotsset{compat=1.18}
\usetikzlibrary{calc, angles, quotes, shapes.geometric, decorations.pathreplacing}
\usepackage{tkz-euclide}
\usepackage[inline]{asymptote}
\usepackage{float}
\usepackage[margin=1in]{geometry}
\usepackage{gensymb}
\usepackage[normalem]{ulem}
\usepackage{hyperref}
\hypersetup{
    colorlinks=true,
    linkcolor=blue,
    filecolor=magenta,      
    urlcolor=cyan,
    pdftitle={Overleaf Example},
    pdfpagemode=FullScreen,
    }
\usepackage{fancyhdr}
\pagestyle{fancy}
\fancyhead[R]{Enoch Yu}
\pagenumbering{gobble}
\usepackage{enumitem}
\newtheorem{theorem}{Theorem}[section]
\newtheorem{lemma}[theorem]{Lemma}
\newtheorem*{lemma*}{Lemma}
\newtheorem{sublemma}{Lemma}[section]
\newtheorem{proposition}{Proposition}
\newtheorem{corollary}{Corollary}[theorem]
\newtheorem{example}{Example}[section]
\newtheorem*{example*}{Example}
\newtheorem{hypothesis}{Hypothesis}[section]
\newtheorem*{hypothesis*}{Hypothesis}
\newenvironment{solution}{\begin{trivlist}\item[]{\bf Solution}}{\qed \end{trivlist}}
\newcommand{\verteq}{\rotatebox{90}{$\;\;=\;\;$}}
\newcommand*\circled[1]{\tikz[baseline=(char.base)]{
            \node[shape=circle,draw,inner sep=1pt] (char) {#1};}}
\newcommand{\triangled}[1]{\tikz[baseline=(char.base)]{
            \node[shape=regular polygon, regular polygon sides=3, draw, inner sep=0.2pt] (char) {#1};}}

\title{Problem Set 24}
\author{Enoch Yu}
\date{June 2025}

\begin{document}

\section*{Problem}
The sequence $\{ a_n \}$ satisfies $a_0 = 0$ and $a_{n + 1} = \frac{8}{5} a_n + \frac{6}{5} \sqrt{4^n - {a_n}^2}$ for $n \ge 0$. Find $a_{10}$.
\begin{solution}

\medskip

\noindent
\textbf{Key Word} $a \cos x + b \sin x = R \cos(x - \alpha)$ where $R = \sqrt{a^2 + b^2}$ and $\tan\alpha = \frac{b}{a}$, Trigonometric Identities

\medskip

\noindent
Let $b_n = \frac{a_n}{2^n}$, $b_{n + 1} = \frac{a_{n + 1}}{2^{n + 1}}$, and ${b_n}^2 = \frac{{a_n}^2}{4^n}$. Therefore, the following equations are true.
\begin{align*}
    2^{n + 1} \cdot b_{n + 1} &= \frac{8}{5} \cdot 2^n \cdot b_n +  \frac{6}{5} \sqrt{4^n - 4^n \cdot {b_n}^2} \\
    2b_{n + 1} &= \frac{8}{5} \cdot b_n +  \frac{6}{5} \sqrt{1 - {b_n}^2} \\
    b_{n + 1} &= \frac{4}{5} \cdot b_n +  \frac{3}{5} \sqrt{1 - {b_n}^2}
\end{align*}

\begin{lemma*}
    $\forall\ i \in \mathbb{Z}, a_i \in \mathbb{R}$.
\end{lemma*}
\begin{proof}
    Inductive steps may lead to the complete proof of the lemma.
    
    \subsection*{Base Case}
    $a_0 = 0$ is given by the problem. Through substitution, it could be inferred that $a_1 = \frac{6}{5}$, which is a real number. Thus, the base case holds true.

    \subsection*{Inductive Steps}
    \begin{hypothesis*}
        If $a_{n + 1} = \frac{8}{5} a_n + \frac{6}{5} \sqrt{4^n - {a_n}^2}$ is a real number, then $a_{n + 2} = \frac{8}{5} a_{n + 1} + \frac{6}{5} \sqrt{4^{n + 1} - {a_{n + 1}}^2}$ is also a real number. 
    \end{hypothesis*}
    \begin{proof}
        \begin{align*}
            a_{n + 2}
            &= \frac{8}{5} a_{n + 1} + \frac{6}{5} \sqrt{4^{n + 1} - {a_{n + 1}}^2} \\
            &= \frac{8}{5} \left( \frac{8}{5} a_n + \frac{6}{5} \sqrt{4^n - {a_n}^2} \right) + \frac{6}{5} \sqrt{4^{n + 1} - \left( \frac{8}{5} a_n + \frac{6}{5} \sqrt{4^n - {a_n}^2} \right)^2}
        \end{align*}
        Because the first term is always real, only the second term could be investigated.
        \[
        \quad \frac{6}{5} \sqrt{4^{n + 1} - \left( \frac{8}{5} a_n + \frac{6}{5} \sqrt{4^n - {a_n}^2} \right)^2}
        \]
        The sign of the expression inside the square root could be checked.
        \begin{align*}
            &\quad\ 4^{n + 1} - \left( \frac{8}{5} a_n + \frac{6}{5} \sqrt{4^n - {a_n}^2} \right)^2 \\
            &= 4^{n + 1} - \left( \frac{64}{25} {a_n}^2 + \frac{96}{25} a_n \sqrt{4^n - {a_n}^2} + \frac{36}{25} (4^n - {a_n}^2) \right) \\
            &= \frac{64}{25} \cdot 4^n - \frac{28}{25} {a_n}^2 - \frac{96}{25} a_n \sqrt{4^n - {a_n}^2}
        \end{align*}
        From the hypothesis, the range for $a_n$ could be found.
        \begin{align*}
            &-2^n \le a_n \le 2^n \\
            &\text{Let $a_n = x$ and $2^n = y$.}
            \\\\
            &\Rightarrow \frac{64}{25} y^2 - \frac{28}{25} x^2 - \frac{96}{25} x \sqrt{y^2 - x^2} \ \text{ where } -y \le x \le y \\
            &\Rightarrow 16 y^2 - 7 x^2 - 24 x \sqrt{y^2 - x^2}
        \end{align*}
        Notice that $-1 \le \frac{x}{y} \le 1$. Therefore, let $\frac{x}{y} = \sin\theta$ for $-\frac{\pi}{2} \le \theta \le \frac{\pi}{2}$. In other words, $x = y \sin\theta$
        \begin{align*}
            &\quad \  16 y^2 - 7 (y \sin\theta)^2 - 24 (y \sin\theta) \sqrt{y^2 - (y \sin\theta)^2} \\
            &=16 y^2 - 7 (y \sin\theta)^2 - 24 (y \sin\theta) \sqrt{y^2 - (y \sin\theta)^2} \\
            &=16 y^2 - 7 y^2 \sin^2\theta - 24 y^2 \sin\theta \cos\theta \\
            &=y^2(16  - 7 \sin^2\theta - 24 \sin\theta \cos\theta)
            \\\\
            &\Rightarrow 16  - 7 \sin^2\theta - 24 \sin\theta \cos\theta \\
            &= 16 - 7 \sin^2\theta - 12 \sin2\theta \\
            &= 16 - \frac{7 - 7\cos2\theta}{2} - 12\sin2\theta \\
            &= \frac{25}{2} + \frac{7\cos2\theta}{2} - 12\sin2\theta \ (-\pi \le 2\theta \le \pi)
        \end{align*}
        Use the fact that $a \cos x + b \sin x = R \cos(x - \alpha)$ where $R = \sqrt{a^2 + b^2}$ and $\tan\alpha = \frac{b}{a}$ could be utilized.
        \begin{align*}
            & \quad \ \frac{25}{2} + \frac{7\cos2\theta}{2} - 12\sin2\theta \\
            &= \frac{25}{2} + \left( \sqrt{\frac{49}{4} + 144} \right) \cos(2\theta - \alpha) \\
            &= \frac{25}{2} + \frac{25}{2} \cos(2\theta - \alpha) \ge 0
        \end{align*}
    \end{proof}
    \noindent
    Therefore, by induction, $a_i$ and $b_i$ must be real number for all whole numbers $i$.
\end{proof}
\noindent
Because the fact that $-1 \le b_n \le 1$ is proven, let $b_n = \sin\theta_n$.
\begin{align*}
    \sin\theta_{n+1}
    &= \frac{4}{5} \sin\theta_n + \frac{3}{5}|\cos\theta_n| \\
    &= \cos\alpha\sin\theta_n + \sin\alpha|\cos\theta_n| \text{ (Where $\alpha$ is the smaller angle in 3-4-5 right triangle.)}
\end{align*}
From the problem, it is evident that $\theta_0 = 0^{\circ}$. 
\begin{align*}
    &1. \cos\theta_n \ge 0 \implies b_{n + 1}=\sin\theta_{n + 1} = \sin(\alpha + \theta_n) \\
    &2. \cos\theta_n < 0 \implies b_{n + 1}=\sin\theta_{n + 1} = \sin(\alpha - \theta_n)
\end{align*}
$b_i$ could further be calculated.
\begin{align*}
    b_0 &= \sin0^{\circ}  &&\theta_0 = 0^{\circ} \\
    b_1 &= \sin\alpha &&\theta_1 = \alpha \\
    b_2 &= \sin2\alpha &&\theta_2 = 2\alpha \\
    b_3 &= \sin3\alpha &&\theta_3 = 3\alpha \Rightarrow \cos3\alpha = 4\cos^3\alpha - 3\cos\alpha < 0 \\
    b_4 &= \sin(-2\alpha) &&\theta_4 = -2\alpha \\
    b_5 &= \sin(-\alpha) &&\theta_5 = -\alpha \\
    b_6 &= \sin0^{\circ} &&\theta_6 = 0^{\circ} \\
    &\qquad \vdots \\
    b_{10} &= \sin(-2\alpha) &&\theta_{10} = -2\alpha \\
\end{align*}
Because $b_n = \frac{a_n}{2^n}$, the case where $n = 10$ could be investigated.
\begin{align*}
    b_{10} &= \frac{a_{10}}{2^{10}} \\
    a_{10} &= b_{10} \cdot 2^{10} \\
    a_{10}
    &= \sin(-2\alpha) \cdot 2^{10} \\
    &= -2^{10}\sin2\alpha \\
    &= -2^{10} \cdot 2\sin\alpha\cos\alpha \\
    &= -2^{10} \cdot 2 \cdot \frac{3}{5} \cdot \frac{4}{5} \\
    &= \boxed{-\frac{24576}{25}} \\
\end{align*}
\end{solution}

\section*{Problem}
Show that if $x + y + z = xyz$, then $\frac{2x}{1-x^2} + \frac{2y}{1-y^2} + \frac{2z}{1-z^2} = \frac{2x}{1-x^2} \cdot \frac{2y}{1-y^2} \cdot \frac{2z}{1-z^2}$.
\begin{proof}
The form $\frac{2x}{1-x^2}$ looks familiar. Therefore, let $x = \tan\alpha$, $y = \tan\beta$, and $z = \tan\gamma$.
\begin{align*}
    \tan\alpha + \tan\beta + \tan\gamma &= \tan\alpha \tan\beta \tan\gamma \\
    \tan(\alpha + \beta + \gamma) &= \frac{\tan\alpha + \tan\beta + \tan\gamma - \tan\alpha \tan\beta \tan\gamma}{1 - (\tan\alpha \tan\beta + \tan\beta \tan\gamma + \tan\gamma \tan\alpha)} = 0
\end{align*}
However, notice that $\alpha + \beta + \gamma = k\pi$ for some integer $k$ since $\tan(\alpha + \beta + \gamma) = 0$. In other words, $\tan(2\alpha + 2\beta + 2\gamma) = 0$.
\begin{align*}
    \tan(2\alpha + 2\beta + 2\gamma) &= 0 \\[0.5em]
    \frac{\tan2\alpha + \tan2\beta + \tan2\gamma - \tan2\alpha \tan2\beta \tan2\gamma}{1 - (\tan2\alpha \tan2\beta + \tan2\beta \tan2\gamma + \tan2\gamma \tan2\alpha)} &= 0 \\[0.5em]
    \tan2\alpha + \tan2\beta + \tan2\gamma - \tan2\alpha \tan2\beta \tan2\gamma &= 0 \\[0.5em]
    \tan2\alpha + \tan2\beta + \tan2\gamma &= \tan2\alpha \tan2\beta \tan2\gamma \\[0.5em]
    \frac{2\tan\alpha}{1 - \tan^2\alpha} + \frac{2\tan\beta}{1 - \tan^2\beta} + \frac{2\tan\gamma}{1 - \tan^2\gamma} &= \frac{2\tan\alpha}{1 - \tan^2\alpha} \cdot \frac{2\tan\beta}{1 - \tan^2\beta} \cdot \frac{2\tan\gamma}{1 - \tan^2\gamma} \\[0.5em]
    \therefore \frac{2x}{1-x^2} + \frac{2y}{1-y^2} + \frac{2z}{1-z^2} &= \frac{2x}{1-x^2} \cdot \frac{2y}{1-y^2} \cdot \frac{2z}{1-z^2}
\end{align*}
\end{proof}

\newpage
\section*{2000 AIME II Problem 15}
Find the least positive integer $n$ such that
\[
\frac{1}{\sin45^{\circ}\sin46^{\circ}} + \frac{1}{\sin47^{\circ}\sin48^{\circ}} + \dots + \frac{1}{\sin133^{\circ}\sin134^{\circ}} = \frac{1}{\sin n^{\circ}}.
\]
\begin{solution}

\medskip

\noindent
\textbf{Key Word} Conjecture and Proof

\medskip

\noindent
First, multiply $\sin n^{\circ}$ on both sides.
\begin{align*}
    \frac{\sin n^\circ}{\sin m^\circ \sin (m+1)^\circ}
    &= \frac{\sin (k+n-k)^\circ}{\sin m^\circ \sin (m+1)^\circ} \\
    &= \frac{\sin (k+n)^\circ \cos k^\circ - \sin k^\circ \cos (k+n)^\circ}{\sin m^\circ \sin (m+1)^\circ}
\end{align*}
Let $k = m$ since $k$ is could be any number.
\begin{align*}
    &\quad \ \frac{\sin (k+n)^\circ \cos k^\circ}{\sin m^\circ \sin (m+1)^\circ} - \frac{\sin k^\circ \cos (k+n)^\circ}{\sin m^\circ \sin (m+1)^\circ} \\[0.5em]
    &=\frac{\sin (m+n)^\circ \cos m^\circ}{\sin m^\circ \sin (m+1)^\circ} - \frac{\sin m^\circ \cos (m+n)^\circ}{\sin m^\circ \sin (m+1)^\circ}
\end{align*}
\begin{lemma*}
    $n$ is equal to $1$.
\end{lemma*}
\begin{proof}
\begin{align*}
    &=\frac{\sin (m+1)^\circ \cos m^\circ}{\sin m^\circ \sin (m+1)^\circ} - \frac{\sin m^\circ \cos (m+1)^\circ}{\sin m^\circ \sin (m+1)^\circ} \\[0.5em]
    &=\frac{\cos m^\circ}{\sin m^\circ} - \frac{\cos (m+1)^\circ}{\sin (m+1)^\circ} \\[0.5em]
    &= \cot m^\circ - \cot (m+1)^\circ
\end{align*}
The sum of all numbers could be written. Moreover, notice that $\cot\alpha + \cot\beta = 0$ if $\alpha + \beta = 180^\circ$.
\begin{align*}
    &\quad \ \cot 45^\circ - \cot 46^\circ + \cot 47^\circ - \cot 48^\circ + \cdots - \cot 132^\circ + \cot 133^\circ - \cot 134^\circ \\
    &= (\cot 45^\circ + \cot 47^\circ + \cdots + \cot 89^\circ + \cot 91^\circ + \dots + \cot 133^\circ) \\
    &\qquad\qquad\qquad\qquad\qquad\qquad - (\cot 46^\circ + \dots + \cot 88^\circ + \cot 90^\circ + \cot 92^\circ + \dots + \cot 134^\circ) \\
    &= \cot 45^\circ - \cot 90^\circ \\
    &= 1
\end{align*}
Because $1 = 1$, the lemma is true.
\end{proof}
\noindent
$n$ could be 1. Moreover, there are no smaller positive integer less than 1 to test. Thus, the least positive integer $n$ that satisfies the given condition is $\boxed{001}$.
\end{solution}
Uploaded a \href{https://artofproblemsolving.com/wiki/index.php/2000_AIME_II_Problems/Problem_15#Solution_4}{new solution} in AOPS!! \\
\includegraphics[scale=0.1]{Screenshot 2025-06-24 at 2.01.21 PM.png}

\newpage
\section*{Problem}
Prove that $\sum_{k=1}^{n} \arctan \frac{1}{2k^2} = \arctan \frac{n}{n+1}$.
\begin{proof}
    Because inverse trigonometric function is given, let $x = \arctan\alpha$ and $y = \arctan\beta$. Therefore, $\tan x = \alpha$ and $\tan y = \beta$.
    \begin{align*}
        \tan(x + y) 
        &= \frac{\tan x + \tan y}{1 - \tan x \tan y} \\
        &= \frac{\alpha + \beta}{1 - \alpha\beta} \\
        \therefore \arctan \left( \frac{\alpha + \beta}{1 - \alpha\beta} \right) &= \arctan\alpha + \arctan\beta
    \end{align*}
    Because the formula for each factor is given and a formula for the sum must be found, inductive steps could be utilized.
    
    \subsection*{Base Case}
    The formula must satisfy for $n=1$.
    \[
    \sum_{k = 1}^{1} \arctan \frac{1}{2k^2} = \arctan \frac{1}{2} = \arctan \frac{1}{1 + 1}
    \]
    The base case is satisfied.

    \subsection*{Induction}
    \begin{lemma*}
        If $\sum_{k=1}^{i} \arctan \frac{1}{2k^2} = \arctan \frac{i}{i+1}$ is true for some natural number $i$, then $\sum_{k=1}^{i+1} \arctan \frac{1}{2k^2} = \arctan \frac{(i+1)}{(i+1)+1}$ is also true.
    \end{lemma*}
    \begin{proof}
        \begin{align*}
            \sum_{k=1}^{i+1} \arctan \frac{1}{2k^2} 
            &= \sum_{k=1}^{i} \arctan \frac{1}{2k^2} + \arctan \frac{1}{2(i+1)^2} \\[0.5em]
            &= \arctan \frac{i}{i+1} + \arctan \frac{1}{2(i+1)^2} \\[0.5em]
            &= \arctan \left( \frac{\frac{i}{i+1} + \frac{1}{2(i+1)^2}}{1 - \frac{i}{i+1} \cdot \frac{1}{2(i+1)^2}} \right) = \arctan \left( \frac{\frac{2(i+1)i + 1}{2(i+1)^2}}{\frac{2(i+1)^3 - i}{2(i+1)^3}} \right) \\[0.5em]
            &= \arctan \left( \frac{\frac{2(i+1)i + 1}{1}}{\frac{2(i+1)^3 - i}{i+1}} \right) = \arctan \left( \frac{(i+1)(2(i+1)i + 1)}{2(i+1)^3 - i} \right) \\[0.5em]
            &= \arctan \left( \frac{(i+1)(2(i+1)i + 1)}{2i^3 + 6i^2 + 5i + 2} \right) = \arctan \left( \frac{(i+1)(2i^2 + 2i + 1)}{2i^3 + 6i^2 + 5i + 2} \right) = \arctan \left( \frac{(i+1)(2i^2 + 2i + 1)}{(i+2)(2i^2 + 2i + 1)} \right) \\[0.5em]
            &= \arctan \left( \frac{i+1}{i+2} \right)
        \end{align*}
    \end{proof}
    \noindent
    By induction, the equation $\sum_{k=1}^{n} \arctan \frac{1}{2k^2} = \arctan \frac{n}{n+1}$ is true.
\end{proof}

\end{document}
