\documentclass{article}
\usepackage{graphicx}
\usepackage{amsmath,amsthm,amssymb}
\usepackage[font=small,labelfont=bf]{caption}
\usepackage{tikz}
\usepackage{pgfplots}
\pgfplotsset{compat=1.18}
\usetikzlibrary{calc, angles, quotes, shapes.geometric, decorations.pathreplacing}
\usepackage{tkz-euclide}
\usepackage[inline]{asymptote}
\usepackage{float}
\usepackage[margin=1in]{geometry}
\usepackage{gensymb}
\usepackage[normalem]{ulem}
\usepackage{hyperref}
\hypersetup{
    colorlinks=true,
    linkcolor=blue,
    filecolor=magenta,      
    urlcolor=cyan,
    pdftitle={Overleaf Example},
    pdfpagemode=FullScreen,
    }
\usepackage{fancyhdr}
\pagestyle{fancy}
\fancyhead[R]{Enoch Yu}
\pagenumbering{gobble}
\usepackage{enumitem}
\newtheorem{theorem}{Theorem}[section]
\newtheorem{lemma}[theorem]{Lemma}
\newtheorem*{lemma*}{Lemma}
\newtheorem{sublemma}{Lemma}[section]
\newtheorem{proposition}{Proposition}
\newtheorem{corollary}{Corollary}[theorem]
\newtheorem{example}{Example}[section]
\newtheorem*{example*}{Example}
\newtheorem{hypothesis}{Hypothesis}[section]
\newtheorem*{hypothesis*}{Hypothesis}
\newenvironment{solution}{\begin{trivlist}\item[]{\bf Solution}}{\qed \end{trivlist}}
\newcommand{\verteq}{\rotatebox{90}{$\;\;=\;\;$}}
\newcommand*\circled[1]{\tikz[baseline=(char.base)]{
            \node[shape=circle,draw,inner sep=1pt] (char) {#1};}}
\newcommand{\triangled}[1]{\tikz[baseline=(char.base)]{
            \node[shape=regular polygon, regular polygon sides=3, draw, inner sep=0.2pt] (char) {#1};}}

\title{Problem Set 27}
\author{Enoch Yu}
\date{June 2025}

\begin{document}

\section*{1996 AIME Problem 10}
Find the smallest positive integer solution to $\tan19x^\circ = \frac{\cos 96^\circ + \sin 96^\circ}{\cos 96^\circ - \sin 96^\circ}$.
\subsection*{Solution I.}
\begin{solution}
\begin{align*}
    \frac{\cos 96^\circ + \sin 96^\circ}{\cos 96^\circ - \sin 96^\circ}
    &= \frac{\cos 96^\circ + \sin 96^\circ}{\cos 96^\circ - \sin 96^\circ} \cdot \frac{\frac{1}{\cos 96^\circ}}{\frac{1}{\cos 96^\circ}} \\[0.5em]
    &= \frac{1 + \tan 96^\circ}{1 - \tan 96^\circ} \\[0.5em]
    &= \frac{\tan 45^\circ + \tan 96^\circ}{1 - \tan 45^\circ \tan 96^\circ} \\[0.5em]
    &= \tan (45+ 96)^\circ \\[0.5em]
    &= \tan 141^\circ
\end{align*}
Therefore, $\tan19x^\circ = \tan 141^\circ$. In other words, $19x = 141 + 180k$ for some integer $k$.
\begin{align*}
    141 + 180k &\equiv 0 \pmod{19} \\
    8 + 9k &\equiv 0 \pmod{19} \\
    9k &\equiv 11 \pmod{19} \\
    9k &\equiv 11, 30, 49, 68, 87, 106, 125, 144 \pmod{19}
\end{align*}
Substituting $k$, the smallest value of $19x$ is $141 + 180 \cdot 16$, which is $\boxed{159}$.
\end{solution}

\subsection*{Solution II.}
\begin{solution}
Notice that $\sin 45^\circ = \cos 45^\circ$. Therefore, all terms in the numerator and the denominator could be multiplied by either $\sin 45^\circ$ or $\cos 45^\circ$.
\begin{align*}
    \frac{\cos 96^\circ + \sin 96^\circ}{\cos 96^\circ - \sin 96^\circ}
    &= \frac{\cos 96^\circ \sin 45^\circ + \sin 96^\circ \cos 45^\circ}{\cos 96^\circ \cos 45^\circ - \sin 96^\circ \sin 45^\circ} \\[0.5em]
    &= \frac{\sin(96 + 45)^\circ}{\cos(96 + 45)^\circ} \\[0.5em]
    &= \tan 141^\circ
\end{align*}
Modular arithmetic from above could be used to obtain answer $\boxed{159}$.
\end{solution}
Uploaded a \href{https://artofproblemsolving.com/wiki/index.php/1996_AIME_Problems/Problem_10#Solution_5}{new solution} in AOPS!! \\
\includegraphics[scale=0.3]{Screenshot 2025-06-30 at 4.56.57 PM.png}

\newpage
\section*{2008 AIME II Problem 8}
Let $a = \frac{\pi}{2008}$. Find the smallest positive integer $n$ such that
\[
2[\cos a \cdot \sin a + \cos 4a \cdot \sin 2a + \cos 9a \cdot \sin 3a + \dots  \cos(n^2a) \cdot \sin(na)]
\]
is an integer.
\begin{solution}
\\\\
\textbf{Key Word} Telescoping
\\\\
Recall that $\cos\alpha \sin\beta = \frac{\sin(\alpha + \beta) - \sin(\alpha - \beta)}{2}$.
\begin{align*}
    &\ \quad 2[\cos a \cdot \sin a + \cos 4a \cdot \sin 2a + \cos 9a \cdot \sin 3a + \dots  \cos(n^2a) \cdot \sin(na)] \\
    &= 2 \cdot \left[ \frac{\sin(a + a) - \sin(a - a)}{2} + \frac{\sin(4a + 2a) - \sin(4a - 2a)}{2} + \dots + \frac{\sin(n^2a + na) - \sin(n^2a - na)}{2} \right] \\
    &= (\sin(a + a) - \sin(a - a)) + (\sin(4a + 2a) - \sin(4a - 2a)) + \dots + (\sin(n^2a + na) - \sin(n^2a - na)) \\
    &= (\sin 2a - \sin 0a) + (\sin 6a - \sin 2a) + \dots + (\sin (n^2a + na) - \sin (n^2a - na)) \\
    &= (\sin 2a + \sin 6a + \sin 12a + \dots + \sin (n^2a + na)) - (\sin 0a + \sin 2a + \sin 6a + \dots + \sin(n^2a - na)) \\
    &= - \sin 0 + \sin (n^2a + na) \\
    &= \sin (n+1)na \\
    &= \sin \frac{n(n+1)\pi}{2008}
\end{align*}
Using the properties of sine functions, notice that $\frac{n(n+1)\pi}{2008}$ must be in the form of $\frac{k\pi}{2}$ for some integer $k$.
\[
\therefore \frac{n(n + 1)}{1004} = k
\]
Notice that $1004 = 2^2 \cdot 251$. $n$ cannot be $250$. Therefore, the smallest positive integer $n$ is $\boxed{251}$.
\end{solution}

\section*{1989 AHSME Problem 28}
Find the sum of the roots of $\tan^2 x - 9\tan x + 1 = 0$ that are between $x = 0$ and $x = 2\pi$ radians.
\begin{solution}
Multiply $\cos^2 x$ on both sides.
\begin{align*}
    \sin^2 x - 9 \sin x \cos x + \cos^2 x &= 0 \\
    1 - 9 \sin x \cos x &= 0 \\
    \sin x \cos x &= \frac{1}{4} \\
    \sin 2x &= \frac{2}{9}
\end{align*}
The possible values for $2x$ are $\alpha, \pi - \alpha, 2\pi + \alpha, 3\pi - \alpha$. Therefore, the sum of all roots are $\boxed{3\pi}$
\end{solution}

\end{document}
