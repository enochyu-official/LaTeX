\documentclass{article}
\usepackage{graphicx}
\usepackage{amsmath,amsthm,amssymb}
\usepackage[font=small,labelfont=bf]{caption}
\usepackage{tikz}
\usepackage{pgfplots}
\pgfplotsset{compat=1.18}
\usetikzlibrary{calc, angles, quotes, shapes.geometric, decorations.pathreplacing}
\usepackage{tkz-euclide}
\usepackage[inline]{asymptote}
\usepackage{float}
\usepackage[margin=1in]{geometry}
\usepackage{gensymb}
\usepackage[normalem]{ulem}
\usepackage{hyperref}
\hypersetup{
    colorlinks=true,
    linkcolor=blue,
    filecolor=magenta,      
    urlcolor=cyan,
    pdftitle={Overleaf Example},
    pdfpagemode=FullScreen,
    }
\usepackage{fancyhdr}
\pagestyle{fancy}
\fancyhead[R]{Enoch Yu}
\pagenumbering{gobble}
\usepackage{enumitem}
\newtheorem{theorem}{Theorem}[section]
\newtheorem{lemma}[theorem]{Lemma}
\newtheorem*{lemma*}{Lemma}
\newtheorem{sublemma}{Lemma}[section]
\newtheorem{proposition}{Proposition}
\newtheorem{corollary}{Corollary}[theorem]
\newtheorem{example}{Example}[section]
\newtheorem*{example*}{Example}
\newenvironment{solution}{\begin{trivlist}\item[]{\bf Solution}}{\qed \end{trivlist}}
\newcommand{\verteq}{\rotatebox{90}{$\;\;=\;\;$}}
\newcommand*\circled[1]{\tikz[baseline=(char.base)]{
            \node[shape=circle,draw,inner sep=1pt] (char) {#1};}}
\newcommand{\triangled}[1]{\tikz[baseline=(char.base)]{
            \node[shape=regular polygon, regular polygon sides=3, draw, inner sep=0.2pt] (char) {#1};}}

\title{Problem Set 29}
\author{Enoch Yu}
\date{July 2025}

\begin{document}

\section*{Problem}
Determine the sum of the angles $A$ an $B$, where $0^\circ \le A, B \le 180^\circ$, and
\[
\sin A + \sin B = \sqrt{\frac{3}{2}}, \quad \cos A + \cos B = \frac{1}{2}.
\]
\begin{solution}
\begin{align*}
    \frac{2 \cdot \sin \left( \frac{A + B}{2} \right) \cdot \cos \left( \frac{A - B}{2} \right)}{2 \cdot \cos \left( \frac{A + B}{2} \right) \cdot \cos \left( \frac{A - B}{2} \right)} &= \frac{\sqrt{\frac{3}{2}}}{\sqrt{\frac{1}{2}}} \\
    \tan \left( \frac{A + B}{2} \right) &= \sqrt{3}
\end{align*}
Therefore, $A + B = \boxed{120^\circ}$.
\end{solution}

\section*{1986 AIME Problem 3}
If $\tan x + \tan y = 25$ and $\cot x + \cot y = 30$, then what is $\tan(x + y)$?
\begin{solution}
Let $a = \tan x$ and $b = \tan y$. Therefore, $a + b = 25$ and $\frac{1}{a} + \frac{1}{b} = 30$. Thus, $ab = \frac{5}{6}$. Because $\tan(x + y) = \frac{a + b}{1 - ab}$, the final answer is $\boxed{150}$.
\end{solution}

\section*{Problem}
Let $A$ and $B$ be acute angles such that $\tan A = \frac{1}{7}$ and $\sin B = \frac{1}{\sqrt{10}}$. Find the degree measure of $A + 2B$ without using a calculator.
\begin{solution}
Because $A$ and $B$ are acute angles, consider a right triangle with sides lengths $1$-$7$-$\sqrt{50}$ for $A$ and $1$-$3$-$\sqrt{10}$ for $B$.
\begin{align*}
    \sin(A + 2B)
    &= \sin A (\cos^2B - \sin^2B) + 2 \sin B \cos B \cos A \\
    &= \frac{1}{\sqrt{50}} \left( \frac{9}{10} - \frac{1}{10} \right) + 2 \cdot \frac{1}{\sqrt{10}} \cdot \frac{3}{\sqrt{10}} \cdot \frac{7}{\sqrt{50}} \\
    &= \frac{50}{10\sqrt{50}} \\
    &= \frac{\sqrt{2}}{2}
\end{align*}
Therefore $A + 2B = \boxed{45^\circ}$ since both $A$ and $B$ are less than $45^\circ$.
\end{solution}

\section*{Problem}
Find all acute angles $\theta$ such that $\sin \theta + \sin 2\theta = \cos \theta + \cos 2\theta$.
\\\\
\begin{minipage}[t]{0.5\textwidth}
\textbf{Solution I.}
\begin{align*}
    2 \cdot \sin \frac{3\theta}{2} \cdot \cos \frac{\theta}{2} &= 2 \cdot \cos \frac{3\theta}{2} \cdot \cos \frac{\theta}{2} \\
    \tan \frac{3\theta}{2} &= 1 \\
    \therefore \theta &= \boxed{30^\circ}
\end{align*}
\end{minipage}
\begin{minipage}[t]{0.5\textwidth}
\textbf{Solution II.}
\begin{align*}
    (\sin\theta - \cos\theta)^2 &= (\cos2\theta - \sin2\theta)^2 \\
    1 - 2\sin\theta\cos\theta &= 1 - 2\cos2\theta\sin2\theta \\
    \sin2\theta &= 2\cos2\theta\sin2\theta \\
    \cos2\theta &= \frac{1}{2} \\
    \theta &= \boxed{30^\circ}
\end{align*}
\end{minipage}

\newpage
\section*{Problem}
Solve the system
\begin{align*}
    2x + yx^2 &= y, \\
    2y + zy^2 &= z, \\
    2z + xz^2 &= x
\end{align*}
for real numbers $x$, $y$, and $z$.
\begin{solution}
Because $x$ can be any number, let $x = \tan\theta$. Notice that $y = \frac{2x}{1 - x^2}$. In other words, $y = \tan2\theta$. Similarly, $z = \frac{2y}{1 - y^2}$. In simpler terms, $z = \tan4\theta$. Continuing, $x = \frac{2z}{1 - z^2}$. Namely, $x = \tan\theta = \tan8\theta$.
\\\\
The fact that $8\theta - \theta = \pi n$ for $n \in \mathbb{Z}$ could be derived. Since $\theta = \frac{\pi n}{7}$, substitution could be used.
\begin{align*}
    x &= \tan\frac{\pi n}{7} \\
    y &= \tan\frac{2\pi n}{7} \\
    z &= \tan\frac{4\pi n}{7}
\end{align*}
Because $\tan$ functions have a period of $\pi$, it is evident that there would be $7$ $0\sim6$ possible answers since for $n \ge 7$, the values will repeat. Thus, the solutions for the system are $\boxed{(0,0,0), \left( \tan\frac{\pi}{7}, \tan\frac{2\pi}{7}, \tan\frac{4\pi}{7} \right),}$ $\boxed{\left( \tan\frac{2\pi}{7}, \tan\frac{4\pi}{7}, \tan\frac{8\pi}{7} \right), \left( \tan\frac{3\pi}{7}, \tan\frac{6\pi}{7}, \tan\frac{12\pi}{7} \right), \left( \tan\frac{4\pi}{7}, \tan\frac{8\pi}{7}, \tan\frac{16\pi}{7} \right), \left( \tan\frac{5\pi}{7}, \tan\frac{10\pi}{7}, \tan\frac{20\pi}{7} \right),}$ $\boxed{\left( \tan\frac{6\pi}{7}, \tan\frac{12\pi}{7}, \tan\frac{24\pi}{7} \right)}$.
\end{solution}

\section*{1995 AIME Problem 7}
Given that
\[
(1 + \sin t)(1 + \cos t) = \frac{5}{4},
\]
find $(1 - \sin t)(1 - \cos t)$.
\begin{solution}
\[
\begin{cases}
    1 + \sin t + \cos t + \sin t \cdot \cos t = \frac{5}{4} \\
    1 - \sin t - \cos t + \sin t \cdot \cos t = x
\end{cases}
\]
Therefore, utilizing both equations will lead to following equations.
\begin{align*}
    1 + \sin t \cdot \cos t &= \frac{5}{8} + \frac{x}{2} \\
    \sin t + \cos t &= \frac{5}{8} - \frac{x}{2}
\end{align*}
Therefore,
\begin{align*}
    1 + 2 \sin t \cos t &= \left( \frac{5}{8} - \frac{x}{2} \right)^2 \\
    \sin t \cdot \cos t &= \frac{\left( \frac{5}{8} - \frac{x}{2} \right)^2 - 1}{2} \\
    1 + \frac{\left( \frac{5}{8} - \frac{x}{2} \right)^2 - 1}{2} &= \frac{5}{8} + \frac{x}{2} \\
    128 + 25 - 40x + 16x^2 - 64 &= 80 + 64x \\
    16x^2 -104x + 9 &= 0 \\
    \therefore x &= \boxed{\frac{13 - 4\sqrt{10}}{4}}
\end{align*}
\end{solution}

\section*{Problem}
Show that $\frac{\sin x + \sin 3x + \sin 5x}{\cos x + \cos 3x + \cos 5x} = \tan 3x$ for all $x$ for which $\tan 3x$ is defined.
\begin{proof}
\begin{align*}
    \frac{2 \sin 3x \cos 2x + \sin 3x}{2 \cos 3x \cos 2x + \cos 3x}
    &= \tan 3x \cdot \left( \frac{2 \cos 2x + 1}{2 \cos 2x + 1} \right) \\
    &= \tan 3x
\end{align*}
\end{proof}

\end{document}
